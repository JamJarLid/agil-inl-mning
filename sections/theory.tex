\section{Theory}
  \subsection{CI/CD}
    Within DevOps, the terms CI/CD are used frequently when discussing the development process. CI stands for Continuous Integration, a practise to merge changes in the codebase back to the main branch as often as possible, in order to reduce information assymetry between developers and avoid integration challenges that can arise when multiple people work on the same feature. The way that continuous integration often is introduced is through pipelines containing automated tests triggered upon push or merge, that check the incoming changes for potential breaks before the code is merged into the main branch.\ref{Atlassian}

  \paragraph{}
    The term CD can have two different meanings, namely Continuous Delivery or Continuous Deployment. Continuous Delivery extends the concept of continuous integration to not only automate the process of testing and integration, to deploying the changes into a testing or production environment after the building stage. By adding this extra step of automation, the development team has the ability to deliver their new product or feature daily, weekly, or however often they want.\ref*{Atlassian}
  
  \paragraph{}
    Continuous Deployment takes the concept of continuous delivery one step further. By utilizing continuous deployment the new product is automatically released to the customer. Through this practice, a company can fully automate the delivery and release process, thereby removing all human interaction with the release pipeline.\ref*{Atlassian}